\documentclass[9pt]{IEEEtran}

\usepackage[english]{babel}
\usepackage{graphicx}
\usepackage{epstopdf}
\usepackage{fancyhdr}
\usepackage{amsmath}
\usepackage{amsthm}
\usepackage{amssymb}
\usepackage{url}
\usepackage{array}
\usepackage{textcomp}
\usepackage{listings}
\usepackage{hyperref}
\usepackage{xcolor}
\usepackage{colortbl}
\usepackage{float}
\usepackage{gensymb}
\usepackage{longtable}
\usepackage{supertabular}
\usepackage{multicol}

\usepackage[utf8x]{inputenc}

\usepackage[T1]{fontenc}
\usepackage{lmodern}
\input{glyphtounicode}
\pdfgentounicode=1

\graphicspath{{./figures/}}
\DeclareGraphicsExtensions{.pdf,.png,.jpg,.eps}

% correct bad hyphenation here
\hyphenation{op-tical net-works semi-conduc-tor trig-gs}

% ============================================================================================

\title{\vspace{0ex}
Mathematics 2, Part 4, Homework 2}


% ============================================================================================

\begin{document}

\maketitle
\section{Duals and duals and duals}
\subsection*{1)}
For the first task, as instructed we can rewrite the (D) problem in the 
standard form: 
\[
\begin{aligned}
\max -b^\top y \\ 
-A^\top y \le -c \\
 y \ge 0
\end{aligned}
\]

Then we can dualize according to the given pattern to get:
\[
\begin{aligned}
\max -c^\top x \\ 
-A^\top x \ge -b \\
 x \ge 0
\end{aligned}
\]
And finally by multiplying with -1, we get back to the (P) problem:
\[
\begin{aligned}
\max c^\top x \\ 
A^\top x \le b \\
 x \ge 0
\end{aligned}
\]

\subsection*{2)}
For this problem we took the (P) and (D) problem from~\cite{mehlhorn2016still}:
\[
\begin{array}{l l}
\centering
\text{(P):} & \text{(D):} \\
\min c^\top x & \max b^\top y \\
Ax = b & A^\top y + s = c \\
x \ge 0 & s \ge 0, \; y \in \mathbb{R}^m
\end{array}
\]
We show that they are dual by starting from the (P) problem. 


\bibliographystyle{plain}  
\bibliography{report}
\end{document}
